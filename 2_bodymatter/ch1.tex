\chapter{A Chapter Title}

Lorem ipsum dolor sit amet, consectetur adipiscing elit. Vestibulum finibus lacinia risus, dapibus venenatis eros eleifend ut. Aenean non dolor varius, tincidunt felis sit amet, aliquet magna. Sed porta ullamcorper tempus. Phasellus sit amet iaculis urna. Nunc et mauris et orci gravida pulvinar non nec elit. Aliquam cursus nibh ut mi pulvinar interdum. Sed elementum, odio eu rhoncus varius, dolor tellus sagittis ante, eget porta nibh eros sit amet magna. Pellentesque a tincidunt nunc. Suspendisse porttitor nisi vitae dui tincidunt, eget rutrum orci pulvinar.

\bigskip
\gls{latex} is written correctly like this \LaTeX. The first word is also linked to the glossary of \gls{latex}. Amazing!

\clearpage
\section{A Section Title} \label{sec:ch1Section}

Morbi mattis feugiat sapien in finibus. Cras imperdiet magna commodo felis rutrum, sed ullamcorper elit tempus. Pellentesque turpis lorem, feugiat sit amet ultrices sed, fringilla vel mauris. Vivamus gravida pulvinar molestie. Curabitur tincidunt, augue eu mattis pretium, orci diam luctus sapien, a auctor elit lectus ac dui. Fusce viverra leo at erat viverra mollis. Sed dignissim libero eu risus vestibulum, ac accumsan libero mattis. In hac habitasse platea dictumst. Proin sollicitudin ligula sed purus accumsan congue. Aliquam ut massa purus. Sed mauris magna, semper non tempus a, placerat a nisl. Aliquam ac volutpat elit, quis suscipit leo. Vestibulum ornare justo sed orci feugiat, ut vestibulum erat sodales. Sed sagittis velit massa, nec laoreet nisl lacinia at.

\begin{figure}[htp]
  \centering
  \includegraphics[width=\textwidth, keepaspectratio]{images/scaredCat.jpg}
  \caption{This is a figure displaying a cat being anxious about \LaTeX. Don`t be. \\\tiny{Photo by Mikhail Vasilyev}}
  \label{fig:scaredCat}
\end{figure}

Praesent auctor, nisl ac tempus fringilla, mauris lectus scelerisque tortor, fringilla feugiat lacus odio eget enim. Praesent eu risus facilisis, hendrerit neque vitae, bibendum mauris. Nullam quis molestie ipsum. Nunc justo ante, lacinia eu tempor eget, faucibus eu risus. Nullam ullamcorper semper nibh at aliquet.

\smallskip
You can add footnotes\footnote{Do. Or do not. There is no try.} and reference figures. Please take a look at figure \ref{fig:scaredCat}.

\subsection{A Sub Section Title}

Maecenas ac nibh cursus, lobortis dolor id, consequat tellus. Proin eu tempor nulla, at luctus ligula. Nam et justo vestibulum, ullamcorper lectus a, bibendum erat. Morbi nulla sem, sodales vitae odio quis, maximus consequat nulla. Aenean leo elit, varius sed mauris non, ultrices maximus augue. Vestibulum molestie enim ac lorem blandit aliquam. Duis ullamcorper nisl sapien, ut volutpat metus imperdiet id. Nullam vel lacinia turpis. Aliquam at lacus tortor.

\bigskip
You can also reference sections. The used labels are up to you to choose. Please take a look at section \ref{sec:ch1Section}.

\subsection{A Sub Section Title}

Pellentesque facilisis risus ligula, a pharetra magna elementum at. Nam at nisi in enim commodo vehicula eu a enim. Aenean venenatis porttitor elit vel sollicitudin. Aliquam venenatis eros non arcu feugiat, quis consectetur diam vehicula. Maecenas ipsum arcu, convallis facilisis massa sit amet, iaculis blandit lacus. Nulla facilisi. Suspendisse vel tincidunt mi, eget feugiat elit. Ut ultrices pharetra diam nec rhoncus. Vivamus aliquam mattis sem, a laoreet neque dignissim ac. Donec sed commodo ante, et dapibus est. Quisque quis sapien vitae lacus placerat vestibulum. Nam blandit ut lacus vitae varius. Nam in tempus massa.

\bigskip
This is a list. Isn`t it amazing?

\begin{itemize}
  \item pellentesque,
  \item facilisis,
  \item risus,
  \item ligula,
  \item et pharetra.
\end{itemize}